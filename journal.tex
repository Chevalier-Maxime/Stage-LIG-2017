% Created 2017-05-09 mar. 14:05
% Intended LaTeX compiler: pdflatex
\documentclass[11pt]{article}
\usepackage[utf8]{inputenc}
\usepackage[T1]{fontenc}
\usepackage{graphicx}
\usepackage{grffile}
\usepackage{longtable}
\usepackage{wrapfig}
\usepackage{rotating}
\usepackage[normalem]{ulem}
\usepackage{amsmath}
\usepackage{textcomp}
\usepackage{amssymb}
\usepackage{capt-of}
\usepackage{hyperref}
\author{Put your name here}
\date{\today}
\title{Journal}
\hypersetup{
 pdfauthor={Put your name here},
 pdftitle={Journal},
 pdfkeywords={},
 pdfsubject={},
 pdfcreator={Emacs 25.2.1 (Org mode 9.0.6)}, 
 pdflang={English}}
\begin{document}

\maketitle
\tableofcontents


\section{2016}
\label{sec:orge331be9}
\subsection{2016-03 March}
\label{sec:orgb54a22a}
\subsubsection{2016-03-07 Monday}
\label{sec:orgd729998}
\begin{enumerate}
\item \href{https://github.com/alegrand/RR\_webinars/blob/master/1\_replicable\_article\_laboratory\_notebook/index.org}{First webinar on reproducible research: litterate programming}
\label{sec:org16e9dad}
\begin{enumerate}
\item Emacs shortcuts
\label{sec:org2a72a70}
Here are a few convenient emacs shortcuts for those that have never
used emacs. In all of the emacs shortcuts, \texttt{C=Ctrl}, \texttt{M=Alt/Esc} and
\texttt{S=Shift}.  Note that you may want to use two hours to follow the emacs
tutorial (\texttt{C-h t}). In the configuration file CUA keys have been
activated and allow you to use classical copy/paste (\texttt{C-c/C-v})
shortcuts. This can be changed from the Options menu.
\begin{itemize}
\item \texttt{C-x C-c} exit
\item \texttt{C-x C-s} save buffer
\item \texttt{C-g} panic mode ;) type this whenever you want to exit an awful
series of shortcuts
\item \texttt{C-Space} start selection marker although selection with shift and
arrows should work as well
\item \texttt{C-l} reposition the screen
\item \texttt{C-\_} (or \texttt{C-z} if CUA keys have been activated)
\item \texttt{C-s} search
\item \texttt{M-\%} replace
\item \texttt{C-x C-h} get the list of emacs shortcuts
\item \texttt{C-c C-h} get the list of emacs shortcuts considering the mode you are
currently using (e.g., C, Lisp, org, \ldots{})
\end{itemize}
There are a bunch of cheatsheets also available out there (e.g.,
\href{http://www.shortcutworld.com/en/linux/Emacs\_23.2.1.html}{this one for emacs} and \href{http://orgmode.org/orgcard.txt}{this one for org-mode} or this \href{http://sachachua.com/blog/wp-content/uploads/2013/05/How-to-Learn-Emacs-v2-Large.png}{graphical one}).
\item Org-mode\hfill{}\textsc{OrgMode}
\label{sec:org9098f78}
Many emacs shortcuts start by \texttt{C-x}. Org-mode's shortcuts generaly
start with \texttt{C-c}.
\begin{itemize}
\item \texttt{Tab} fold/unfold
\item \texttt{C-c c} capture (finish capturing with \texttt{C-c C-c}, this is explained on
the top of the buffer that just opened)
\item \texttt{C-c C-c} do something useful here (tag, execute, \ldots{})
\item \texttt{C-c C-o} open link
\item \texttt{C-c C-t} switch todo
\item \texttt{C-c C-e} export
\item \texttt{M-Enter} new item/section
\item \texttt{C-c a} agenda (try the \texttt{L} option)
\item \texttt{C-c C-a} attach files
\item \texttt{C-c C-d} set a deadl1ine (use \texttt{S-arrows} to navigate in the dates)
\item \texttt{A-arrows} move subtree (add shift for the whole subtree)
\end{itemize}
\item Org-mode Babel (for literate programming)\hfill{}\textsc{OrgMode}
\label{sec:orgf5595a8}
\begin{itemize}
\item \texttt{<s + tab} template for source bloc. You can easily adapt it to get this:
\begin{verbatim}
#+BEGIN_SRC sh
ls
#+END_SRC
\end{verbatim}
Now if you \texttt{C-c C-c}, it will execute the block.
\begin{verbatim}
  #+RESULTS:
  | #journal.org# |
  | journal.html  |
  | journal.org   |
  | journal.org~  |
\end{verbatim}

\item Source blocks have many options (formatting, arguments, names,
sessions,\ldots{}), which is why I have my own shortcuts \texttt{<b + tab} bash
block (or \texttt{B} for sessions).
\begin{verbatim}
  #+begin_src sh :results output :exports both
  ls /tmp/*201*.pdf
  #+end_src

  #+RESULTS:
  : /tmp/2015_02_bordeaux_otl_tutorial.pdf
  : /tmp/2015-ASPLOS.pdf
  : /tmp/2015-Europar-Threadmap.pdf
  : /tmp/europar2016-1.pdf
  : /tmp/europar2016.pdf
  : /tmp/M2-PDES-planning-examens-janvier2016.pdf
\end{verbatim}
\item I have defined many such templates in my configuration. You can
give a try to \texttt{<r}, \texttt{<R}, \texttt{<RR}, \texttt{<g}, \texttt{<p}, \texttt{<P}, \texttt{<m} \ldots{}
\item Some of these templates are not specific to babel: e.g., \texttt{<h}, \texttt{<l},
\texttt{<L}, \texttt{<c}, \texttt{<e}, \ldots{}
\end{itemize}
\item In case you want to play with ipython on a recent debian\hfill{}\textsc{Python}
\label{sec:orgcd867a7}
Here is what you should install:
\begin{verbatim}
sudo apt-get install python3-pip ipython3 ipython3-notebook python3-numpy python3-matplotlib
\end{verbatim}

The ipython notebook can then be run with the following command:
\begin{verbatim}
ipython3 notebook
\end{verbatim}

The latest version of this notebook is called \href{http://jupyter.org/}{Jupyter} and is polyglot
like babel. Playing with it is easy as it's deployed on the cloud but
as I'm not a python expert I'm not sure to know how to deploy it locally.
\item In case you want to play with R/knitR/rstudio:\hfill{}\textsc{R}
\label{sec:org6b2e4f0}
Here is what you should install on debian:
\begin{verbatim}
sudo apt-get install r-base r-cran-ggplot2
\end{verbatim}

Rstudio and knitr are unfortunately not packaged within debian so the
easiest is to download the corresponding debian package on the \href{http://www.rstudio.com/ide/download/desktop}{Rstudio
webpage} and then to install it manually (depending on when you do
this, you can obviously change the version number).
\begin{verbatim}
wget https://download1.rstudio.org/rstudio-0.99.887-amd64.deb
sudo dpkg -i rstudio-0.99.887-amd64.deb
sudo apt-get -f install # to fix possibly missing dependencies
\end{verbatim}
You will also need to install knitr. To this end, you should simply
run R (or Rstudio) and use the following command.
\begin{verbatim}
install.packages("knitr")
\end{verbatim}
If \texttt{r-cran-ggplot2} could not be installed for some reason, you can also
install it through R by doing:
\begin{verbatim}
install.packages("ggplot2")
\end{verbatim}

As you will experience, knitr is polyglot but not Rstudio, which
makes its use not as fluid when using other languages than R.
\end{enumerate}
\end{enumerate}
\section{2017}
\label{sec:orgcaff67a}
\subsection{2017-05 mai}
\label{sec:org52dea03}
\subsubsection{2017-05-09 mardi}
\label{sec:org58ae459}
\begin{enumerate}
\item Premiers pas avec org-mode
\label{sec:org58ccbe8}
\item Réunion avec Florence et Arnaud
\label{sec:org0e9e98a}
\begin{enumerate}
\item {\bfseries\sffamily TODO} Mettre en place mon cahier de laboratoire [1/3]
\label{sec:org1c6f0bf}
\url{http://www.google.com}
\begin{itemize}
\item $\boxminus$ github
\item $\boxtimes$ org-mode
\item $\square$ intégrer le mail
\end{itemize}

Entered on \textit{[2017-05-09 mar. 13:45]}
\item 
\label{sec:org29681dd}
\end{enumerate}
\end{enumerate}
\end{document}
